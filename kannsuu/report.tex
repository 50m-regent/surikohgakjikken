\documentclass[a4j, titlepage]{jsarticle}
\usepackage[dvipdfmx]{graphicx}
\usepackage{amsmath,ascmac,amsthm, amssymb}
\usepackage{amsfonts,latexsym,mathtools}
\usepackage{bm}
\usepackage{algorithm,algorithmic}
\usepackage{listings}
\usepackage{empheq}
\lstset{
    language={C},
    basicstyle={\small\ttfamily},
    identifierstyle={\small},
    commentstyle={\small\itshape},
    keywordstyle={\small\bfseries},
    ndkeywordstyle={\small},
    stringstyle={\small\ttfamily},
    frame={tb},
    breaklines=true,
    columns=[l]{fullflexible},
    numbers=left,
    xrightmargin=0zw,
    xleftmargin=3zw,
    numberstyle={\scriptsize},
    stepnumber=1,
    numbersep=1zw,
    lineskip=-0.5ex
}
\renewcommand{\lstlistingname}{コード}
\renewcommand{\lstlistlistingname}{コード目次}

\numberwithin{equation}{section}
%\setcounter{tocdepth}{3}

\begin{document}
\begin{titlepage}
    \begin{center}
        {\Large 令和四年度後期 数理工学実験 課題レポート}

        \vspace*{180truept}

        {\Huge 関数の補間と数値積分}

        \vspace{160truept}

        {\Large 情報学科2回生 平田蓮}

        \vspace{10truept}

        {\large 学生番号: 1029342830}

        \vspace{60truept}

        {\large 実験日: 11月8, 14, 15, 28日}

        \vspace{10truept}

        {\large 実験場所: 京都大学工学部総合校舎数理計算機室}

        \vspace{60truept}

        {\large 12月5日 提出}
    \end{center}
\end{titlepage}

\tableofcontents
\clearpage

\section{目的}
    

\section{原理}
    

\section{課題}
    

\newpage
\addcontentsline{toc}{section}{参考文献}
\begin{thebibliography}{99}
    \bibitem{text}{
        実験演習ワーキンググループ、``数理工学実験 2022年度版''、京都大学工学部情報学科数理工学コース (2022)
    }
\end{thebibliography}
\end{document}