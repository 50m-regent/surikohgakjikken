\documentclass[a4j, titlepage]{jsarticle}
\usepackage[dvipdfmx]{graphicx}
\usepackage{amsmath,ascmac,amsthm}
\usepackage{bm}
\usepackage{algorithm,algorithmic}
\usepackage{listings}
\lstset{
    language={C},
    basicstyle={\small\ttfamily},
    identifierstyle={\small},
    commentstyle={\small\itshape},
    keywordstyle={\small\bfseries},
    ndkeywordstyle={\small},
    stringstyle={\small\ttfamily},
    frame={tb},
    breaklines=true,
    columns=[l]{fullflexible},
    numbers=left,
    xrightmargin=0zw,
    xleftmargin=3zw,
    numberstyle={\scriptsize},
    stepnumber=1,
    numbersep=1zw,
    lineskip=-0.5ex
}
\renewcommand{\lstlistingname}{コード}
\renewcommand{\lstlistlistingname}{コード目次}

\title{
    数理工学実験レポート \\
    常微分方程式の数値解法 \\
    \large{実験日: 10月11, 17, 18, 24日} \\
    \large{実験場所: 工学部総合校舎 数理計算機室}
}
\author{情報学科2回生 平田 蓮(学籍番号 1029342830)}

\begin{document}
\maketitle

\section{目的}
    多くの数理モデルには、微分方程式が現れる。
    これの解を解析的に得ることは一般的には難しいため、
    数値計算でその近似解求めるためのアルゴリズムを学習する。

\section{原理}
    本章では、続く課題で用いるアルゴリズムについて述べる。

    \subsection{オイラー法}
        以下の関数を考える。

        \begin{equation}
            \frac{\mathrm{d}u}{\mathrm{dt}}(t) = f(t, u(t)), \ u(0) = u_0 \label{equ:diff}
        \end{equation}

        この式の両辺を$t$について、$[t_a, t_b]$の範囲で積分すると、
        
        \begin{equation*}
            u(t_b) - u(t_a) = \int^{t_b}_{t_a}f(t, u(t))\mathrm{d}t
        \end{equation*}

        を得る。$\Delta t = t_b - t_a$とすると、
        範囲内の積分を幅$\Delta t$、高さ$f(t_a, u(t_a))$の
        長方形の面積で近似でき、

        \begin{alignat*}{2}
            & u(t_b) - u(t_a) & \ \approx \ & f(t_a, u(t_a))\Delta t \\
            \rightarrow \ & u(t_b) & \ \approx \ & u(t_a) + f(t_a, u(t_a))\Delta t
        \end{alignat*}

        と書ける。ここで$(a, b)$を$(n, n + 1)$に置き換え、
        さらに$u(t_n)$を$u_n$に書き換えると、

        \begin{equation}
            u_{n + 1} = u_n + f(t_n, u_n)\Delta t
        \end{equation}

        を得る。この式を$n$に対して繰り返し適用することで、
        式(\ref{equ:diff})を満たす関数を数値計算することができる。

\begin{thebibliography}{99}
    \bibitem{text}{
        実験演習ワーキンググループ, ``数理工学実験 2022年度版'', 京都大学工学部情報学科数理工学コース
    }
\end{thebibliography}
\end{document}